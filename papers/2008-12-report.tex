\documentclass[11pt]{article}
\usepackage{url}

\begin{document}

Work on machine-translation evaluation measures continued as Kahn,
Ostendorf and Roark developed Expected Dependency Pair Match (EDPM).  

EDPM uses a popular PCFG syntactic parser \cite{charniak-johnson:2005:ACL}
to extract expected counts of dependency structures from hypothesis
and reference translations and score them. 
%
EDPM is based on the intuitive and difficult-to-game $F$ measure, where
the tokens to be matched are drawn from 1-grams, 2-gram, and in- and
out-bound dependency links.  In this respect, it extends the
dependency-scoring strategies of \cite{owczarzak07labelleddepseval} but does so
with a widely-used and publicly available PCFG parser and head-finding
rules.  Further, EDPM incorporates string-only features and uses
the parser's own confidence to predict the (hidden) dependency structure.
%
EDPM's free parameters thus include the number of parses $n$ used in the
calculation, the classes of tokens to be extracted for the F-measure,
and the degree in which to trust the parser's confidence distribution
over the $n$-best list.

% [experiments on some other corpora]

In experiments over the Multiple Translation Chinese Corpus \cite{LDC03MTC2,LDC06MTC4}, we explored a variety of measures in the EDPM family and
selected the variant that best-correlated with human judgments of
fluency and adequacy per sentence.  Over this corpus, the selected
variant of EDPM correlated at $r=0.240$, much better than popular
measures TER ($r=-0.173$) and (add-one-smoothed) BLEU$_4$ ($r=0.218$).

% [differences from other popular measures] 

The utility of this chosen variant was further tested in predicting
the difference in human-targeted translation edit rate ($\Delta$HTER)
between two translations of the same source from the unsequestered
GALE 2.5 \cite{darpa08gale} Arabic-to-English and Chinese-to-English task.
EDPM's per-document correlation with $\Delta$HTER over this corpus was
better ($r=-0.47$) than BLEU$_4$ ($r=-0.32$) or TER ($r=0.39$), with
similar effects on per-sentence correlations.

% [competition]

EDPM was submitted to the 2008 NIST MetricsMATR MT-evaluation
competition. In this competition, EDPM performed the best overall on
document-level HTER correlation, and within the confidence of the
best-performance on all HTER measures (and nearly all other
human-derived measures as well).
XC
Future work on EDPM will include the ability to change parsers (making
it much easier to apply it to new domains or languages), partial-match
scoring (using synonyms and other partial-match techniques) and
exploration of other dependency graph-fragments.

\bibliographystyle{alpha}
\bibliography{2008-12-report}

\end{document}
